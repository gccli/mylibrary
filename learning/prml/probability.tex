\documentclass[a4paper]{article}
\usepackage{xeCJK}
\setCJKmainfont[BoldFont={WenQuanYi Zen Hei}, ItalicFont={WenQuanYi Zen Hei}]{WenQuanYi Zen Hei Mono}
%\usepackage{xeCJK}
%\setCJKmainfont{STFangsong}
\usepackage{geometry}
\usepackage{mathtools}
\usepackage[sc,osf,slantedGreek]{mathpazo}
\begin{document}
\title{Probability Distributions}
\author{}
\maketitle

The SVM is a decision machine and so does not provide posterior probabilities.
\section{二元变量(Binary Variables)}
$x \in \{0,1\}, p(x=1|\mu) = \mu, p(x=0|\mu) = 1-\mu$, Bernoulli distribution $Bern(x|\mu) = \mu^x(1-\mu)^{1-x}$
data set $D = {x_1,...,x_n}$ of observed values of x. likehood function
\begin{align}
p(D|\mu) = \prod_{n=1}^N p(x_n|\mu) = \prod_{n=1}^N \mu^{x_n}(1-\mu)^{1-x_n}\\
ln p(D|\mu) = \sum_{n=1}^N \{ x_nln\mu + (1-x_n)ln(1-\mu) \}
\end{align}
binomial distribution
\begin{align}
  Bin(m|N,\mu) = {N \choose m} \mu^m (1-\mu)^{N-m}
\end{align}

\subsection{Beta Distribution}

Introduce a prior distribution p(μ)
\begin{align}
  Beta(\mu|a,b) = \frac{ \Gamma(a+b) } {\Gamma(a)\Gamma(b)} \mu^{a-1} (1-\mu)^{b-1}\\
  \Gamma(t) = \int_0^{inf} x^{t-1}e^{-x} dx\\
  \int_0^1 Beta(\mu|a,b) d\mu = 1\\
  \mathbb{E}[\mu] = \frac{a}{a+b}\\
  var[\mu] = \frac{ab}{(a+b)^2(a+b+1)}
\end{align}

Posterior distribution of $\mu$ is now obtained by multiplying the beta prior by the binomial likelihood function

\begin{align*}
  p(\mu|m,l,a,b) &\propto \mu^{m+a-1} (1-\mu)^{l+b-1}\\
  p(\mu|m,l,a,b) &=  \frac{ \Gamma(m+a+l+b) } {\Gamma(m+a)\Gamma(l+b)} \mu^{m+a-1} (1-\mu)^{l+b-1}
\end{align*}
a and b in the prior as an \textit{effective number of observations} of x = 1 and x = 0

\section{Multinomial Variables}
用参数$\mu_k$表示$x_k=1$的概率,$\mathbf{x}$的分布如下:
\begin{align*}
  \mathbf{x} = (0,0,1,0,0,0)^T\\
  p(\mathbf{x}|\mu) = \prod_{k=1}^K\mu_k^{x_k}
\end{align*}
$N$个独立观测值$\mathbf{x_1},...,\mathbf{x_n}$的数据集$D$
\begin{align*}
  p(D|\mu) = \prod_{n=1}^N  \prod_{k=1}^K \mu_k^{x_{nk}} = \prod_{k=1}^K \mu_k^{m_k}\\
  m_k = \sum_n x_{nk}
\end{align*}
$m_k$表示观测到$x_k=1$的次数,这被称为这个分布的充分统计量(sufficient statistics)
多项式分布(multinomial distribution)
\begin{align*}
  Mult(m_1,m_2,...,m_K|\mu,N) = {N \choose m_1m_2...m_K} \prod_{k=1}^K \mu_k^{m_k}\\
  \sum_{k=1}^K m_k = N
\end{align*}
共轭先验为:
\begin{align*}
  Dir(\mu|\alpha) &= \frac{\Gamma(\alpha_0)}{\Gamma(\alpha_1)...\Gamma(\alpha_K)} \prod_{k=1}^K \mu_k^{m_k}\\
\alpha_0 &= \sum \alpha_k
\end{align*}
似然函数乘以先验,得到参数${\mu_k}$的后验概率
\begin{align*}
  p(\mu|D,\alpha) = Dir(\mu|\alpha+\mathbf{m}) = \frac{\Gamma(\alpha_0+N)}{\Gamma(\alpha_1+m_1)...\Gamma(\alpha_K+m_K)} \prod_{k=1}^K \mu_k^{\alpha_k+m_k-1}
\end{align*}
$\alpha_k$看作$x_k=1$的有效观测数。

\section{高斯分布}
\begin{align*}
  \mathcal{N}(x|\mu,\sigma^2) &= \frac{1}{(2\pi\sigma^2)^{1/2}} exp \{-\frac{1}{2\sigma^2} (x-\mu)^2\}\\
  \mathcal{N}(\mathbf{x}|\boldsymbol{\mu},\boldsymbol{\Sigma}) &= \frac{1}{(2\pi)^{D/2}} \frac{1}{|\Sigma|^{1/2}} exp \{-\frac{1}{2} (x-\mu)^T \Sigma^{-1} (x-\mu) \}\\
  \Delta^2 &= (x-\mu)^T \Sigma^{-1} (x-\mu)
\end{align*}

\subsection{Conditional Gaussian distributions}
\begin{align*}
  \boldsymbol{x} =
    \begin{pmatrix} \boldsymbol{x}_a\\ \boldsymbol{x}_b \end{pmatrix},
  \boldsymbol{\mu} =
    \begin{pmatrix} \boldsymbol{\mu}_a\\ \boldsymbol{\mu}_b \end{pmatrix},
  \boldsymbol{\Sigma} =
    \begin{pmatrix}
      \boldsymbol{\Sigma}_{aa} & \boldsymbol{\Sigma}_{ab}\\
      \boldsymbol{\Sigma}_{ba} & \boldsymbol{\Sigma}_{bb}
    \end{pmatrix}
\end{align*}
协方差矩阵的对称性$\Sigma = \Sigma^T$表明$\Sigma_{aa}, \Sigma_{bb}$也是对称的,$\Sigma_{ab}^T=\Sigma_{ba}$.引入精度矩阵:$\Lambda = \Sigma^{-1}$
\begin{align*}
  \boldsymbol{\Lambda} =
    \begin{pmatrix}
      \boldsymbol{\Lambda}_{aa} & \boldsymbol{\Lambda}_{ab}\\
      \boldsymbol{\Lambda}_{ba} & \boldsymbol{\Lambda}_{bb}
    \end{pmatrix}
\end{align*}

寻找条件概率$p(\boldsymbol{x}_a|\boldsymbol{x}_b)$,用精度矩阵展开二次型$\Delta^2$
$p(\boldsymbol{x}_a|\boldsymbol{x}_b)$也是高斯分布,确定其均值与协方差
选出与$x_a$相关的二阶项,得到方差
\begin{align*}
  -\frac{1}{2} \boldsymbol{x}_a^T \boldsymbol{\Lambda}_{aa} \boldsymbol{x}_a
  \Sigma_{a|b} = \Lambda_{aa}^{-1}
\end{align*}
选出与$x_a$相关的线性项,all of the terms that are linear in $x_a$
\begin{align*}
  x_a^T\{ \Lambda_{aa}\mu_a - \Lambda_{ab}(x_b-\mu_b) \}\\
  \mu_{a|b} &= \Sigma_{a|b} \{ \Lambda_{aa}\mu_a - \Lambda_{ab}(x_b-\mu_b) \}\\
         &=  \mu_a - \Lambda_{aa}^{-1}\Lambda_{ab}(x_b - \mu_b)
\end{align*}

\subsection{Marginal Gaussian distributions}
$p(x_a) = \int p(x_a,x_b) dx_b$
注意力集中于联合分布的指数项的二次型
\begin{align*}
  \mathbb{E}[x_a] &= \mu_a\\
  var[x_a] &= \Sigma_{aa}
\end{align*}


分块高斯的边缘分布条件分布总结,给定联合高斯分布$\mathcal{N}(\mathbf{x}|\boldsymbol{\mu},\boldsymbol{\Sigma})$
\begin{align*}
  \boldsymbol{x} =
    \begin{pmatrix} \boldsymbol{x}_a\\ \boldsymbol{x}_b \end{pmatrix},
  \boldsymbol{\mu} =
    \begin{pmatrix} \boldsymbol{\mu}_a\\ \boldsymbol{\mu}_b \end{pmatrix}\\
  \boldsymbol{\Sigma} =
    \begin{pmatrix}
      \boldsymbol{\Sigma}_{aa} & \boldsymbol{\Sigma}_{ab}\\
      \boldsymbol{\Sigma}_{ba} & \boldsymbol{\Sigma}_{bb}
    \end{pmatrix},
  \boldsymbol{\Lambda} =
    \begin{pmatrix}
      \boldsymbol{\Lambda}_{aa} & \boldsymbol{\Lambda}_{ab}\\
      \boldsymbol{\Lambda}_{ba} & \boldsymbol{\Lambda}_{bb}
    \end{pmatrix}
\end{align*}
条件概率分布
\begin{align*}
  p(x_a|x_b) &= \mathcal{N}(\mathbf{x_a}|\boldsymbol{\mu}_{a|b},\boldsymbol{\Lambda}_{aa}^{-1})\\
  \mu_{a|b}  &=  \mu_a - \Lambda_{aa}^{-1}\Lambda_{ab}(x_b - \mu_b)
\end{align*}
边缘概率分布
\begin{align*}
  p(x_a) &= \mathcal{N}(\mathbf{x_a}|\boldsymbol{\mu}_a,\boldsymbol{\Sigma}_{aa})
\end{align*}



\end{document}
