\documentclass[a4paper]{article}
\usepackage{geometry}
\usepackage{mathtools}
\usepackage[sc,osf,slantedGreek]{mathpazo}
\begin{document}

\section{Bayesian Linear Regression}

\subsection{Parameter distribution}

conjugate prior over the model parameters $ \mathbf{w} $

\[
p(\mathbf{w}) = \mathcal{N}(\mathbf{w} | \mathbf{m}_0, \mathbf{S}_0)
\]
having mean $ \mathbf{m}_0 $ and convariance $ \mathbf{S}_0 $
posterior distribution

\begin{align}
p(\mathbf{w}|\mathbf{t}) = \mathcal{N}(\mathbf{w} | \mathbf{m}_N, \mathbf{S}_N)
\\
\mathbf{m}_N = \mathbf{S}_N(\mathbf{S}_0^{-1} \mathbf{m}_0 + \beta \Phi^T \mathbf{t}))
\\
\mathbf{S}_N^{-1} = \mathbf{S}_0^{-1} + \beta \Phi^T \Phi
\end{align}

Specifically, we consider a zero-mean
isotropic Gaussian governed by a single precision parameter α so that

\begin{align}
p(\mathbf{w}|\alpha) = \mathcal{N}(\mathbf{w} | \mathbf{0},
  \alpha^{-1} \mathit{\mathbf{I}})
\\
\mathbf{m}_N = \beta \mathbf{S}_N \Phi^T \mathbf{t}
\\
\mathbf{S}_N^{-1} =  \alpha \mathit{\mathbf{I}} + \beta \Phi^T \Phi
\end{align}


\subsection{Predictive distribution}

\textit{predictive distribution}
\[
p(t | \mathbf{t}, \alpha, \beta) = \int p(t | \mathbf{w},
\beta)p(\mathbf(w)|\mathbf{t}, \alpha, \beta) d\mathbf{w}
\]
$ \mathbf{t} $ is the vector of target values from the training set,

\[
p(t | \mathbf{x}, \mathbf{t}, \alpha, \beta) = \mathcal{N}(t |
\mathbf{m}_N^T\Phi(\mathbf{x}),\sigma_N^2(\mathbf{x}))
\]

the variance $ \sigma_N^2(\mathbf{x}) $ of the predictive distribution is given by
\[
\sigma_N^2(\mathbf{x}) = \frac{1}{\beta} + \Phi(\mathbf{x})^T\mathbf{M}_N\Phi(\mathbf{x})
\]

\subsection{Equivalent kernel}
predictive mean
\begin{align}
y(\mathbf{x},\mathbf{m}_N) = \mathbf{m}_N^T \Phi(\mathbf{x}) =
\displaystyle\sum_{n=1}^{N} \beta \Phi(\mathbf{x})^T \mathcal{S}_N \Phi(\mathbf{x_n}) t_n
\end{align}
the mean of the predictive distribution at a point
$ \mathbf{x} $ is given by a linear combination of the training set target
variables $ t_n $



\begin{align}
y(\mathbf{x},\mathbf{m}_N) =
\displaystyle\sum_{n=1}^{N} k(\mathbf{x},\mathbf{x}_n) t_n
\\
 k(\mathbf{x},\mathbf{x}') = \beta \phi(\mathbf{x})^T \mathbf{S}_N \phi(\mathbf{x'})
\end{align}
is known as the \textit{smoother matrix} or the \textit{equivalent
  kernel}. Regression functions, such
as this, which make predictions by taking linear combinations of the training set
target values are known as \textit{linear smoothers}. Intuitively, it seems reasonable
that we should weight local evidence more strongly than distant evidence.

Instead of introducing a set of basis
functions, which implicitly determines an equivalent kernel, we can instead define
a localized kernel directly and use this to make predictions for new input vectors x,
given the observed training set. This leads to a practical framework for regression
(and classification) called \textit{Gaussian processes}.
We have seen that the effective kernel defines the weights by which the training
set target values are combined in order to make a prediction at a new value of x, and
it can be shown that these weights sum to one, in other words
\begin{align}
\displaystyle\sum_{n=1}^{N} k(\mathbf{x},\mathbf{x}_n) = 1
\end{align}
Finally, we note that the equivalent kernel (3.62) satisfies an
important property shared by kernel functions in general, namely that
it can be expressed in the form an inner product with respect to a
vector $ \psi(\mathbf{x} $ of nonlinear functions, so that
\begin{align}
k(\mathbf{x},\mathbf{z}) = \psi(\mathbf{x})^T\psi(\mathbf{x})
\\
\psi(\mathbf{x}) = \beta^{1/2} \mathbf{S}_N^{1/2} \phi(\mathbf{x})
\end{align}


\end{document}
