\documentclass[a4paper]{article}
\usepackage{xeCJK}
\setCJKmainfont{STKaiti}
\usepackage{amsmath}
\usepackage{geometry}
\usepackage{mathtools}
\usepackage{mathrsfs}
\usepackage[sc,osf,slantedGreek]{mathpazo}
\begin{document}
\title{Gaussian Process classification(GPC)}
\author{}
\maketitle

Supervised learning can be divided into regression and classification problems. outputs for classification are \emph{discrete class labels}, regression is concerned with the \emph{prediction of continuous quantities}.

\section{Regression}
One can think of a Gaussian process as defining a distribution over functions, and inference taking place directly in the space of functions, the \emph{function-space view}

\subsection{Weight-space View}
Bayesian treatment of the linear model. Enhancement to this class of models by projecting the inputs into a high-dimensional \emph{feature space} and applying the linear model.

\textbf{Training set} $\mathcal{D}$ of $n$ observations, $\mathcal{D} = \{(\mathbf{x}_i, y_i)|i = 1,...,n\}$, $\mathbf{x}$ denotes an input vector (covariates) of dimension $D$ and $y$ denotes a scalar output or target (dependent variable). the column vector inputs for all $n$ cases are aggregated in the $D \times n$ \emph{design matrix} $X$, $\mathcal{D} = \{X,\mathbf{y}\}$

\subsubsection{The Standard Linear Model}

\begin{align}
  f(\mathbf{x}) = \mathbf{x}^T\mathbf{w},\;y = f(\mathbf{x}) + \epsilon
\end{align}

\begin{align}
  \epsilon \sim \mathcal{N}(0,\sigma_n^2)
\end{align}

\emph{likelihood}, the probability density of the observations given the parameters, which is factored over cases in the training set
\begin{align}
  p(\mathbf{y}|X,\mathbf{w}) = \prod p(y_i|\mathbf{x}_i,\mathbf{w}) = \mathcal{N}(X^T\mathbf{w},\sigma_n^2 I)
\end{align}

\subsubsection{Projections of Inputs into Feature Space}

first project the inputs into some high dimensional space using a set of \emph{basis functions} and then apply the linear model in this space instead of directly on the inputs themselves. e.g. $\phi(x) = (1, x, x^2, x^3,...)^T$, \emph{the model is still linear in the parameters}. Now the model is
\begin{align}
  f(x) = \phi(x)^T \mathbf{w}
\end{align}
\emph{disign matrix} $\Phi(X)$ is substituted for $X$.

predictive distribution


\subsection{Function-space View}

\textbf{Definition} A Gaussian process is \emph{a collection of random variables, any finite number of which have a joint Gaussian distribution}.
A Gaussian process $f(\mathbf{x})$ is completely specified by its mean function $m(\mathbf{x})$ and covariance function $k(\mathbf{x},\mathbf{x}')$.
\begin{align*}
  m(\mathbf{x}) &= \mathbb{E}[f(\mathbf{x})]\\
  k(\mathbf{x},\mathbf{x}') &= \mathbb{E}[f(\mathbf{x}-m(\mathbf{x}))\: f(\mathbf{x}'-m(\mathbf{x}'))]
\end{align*}
Gaussian process as
\begin{align}
  f(\mathbf{x}) \sim \mathscr{G} \mathcal{P}(m(\mathbf{x}), k(\mathbf{x}, \mathbf{x}'))
\end{align}

A simple example of a Gaussian process can be obtained from our Bayesian linear regression model$f(x) = \phi(x)^T\mathbf{w}$ with prior $\mathbf{w} \sim \mathcal{N}(0,\Sigma_p)$

\begin{align*}
  \mathbb{E}[f(x)] &= \phi(\mathbf{x})^T \mathbb{E}[w] = 0\\
  \mathbb{E}[f(x)f'(x)] &= \phi(x)^T \Sigma_p \phi(x')
\end{align*}

\emph{squared exponential}
\begin{align*}
  cov[f(x_p), f(x_q)] = k(x_p,x_q) = exp(-\frac{1}{2} |x_p - x_q|^2)
\end{align*}

\begin{align}
  f_* \sim \mathcal{N} (0, K(X_*, X_*))
\end{align}

\paragraph{Prediction with Noise-free Observations}
Consider the simple special case where the observations are noise free $\{(\mathbf{x_i},f_i)| i=1,...,n\}$
The joint distribution of the training outputs $f$ and the test outputs $f_*$ is
\begin{align}
  \begin{bmatrix}
    f\\
    f_*
  \end{bmatrix}
  \sim \mathcal{N}(0,
  \begin{bmatrix}
    K(X,X) & K(X,X_*)\\
    K(X_*,X) & K(X_*,X_*)\\
  \end{bmatrix})
\end{align}
$K(X, X_*)$ denotes the $n \times n_*$ matrix of the covariances evaluated at all pairs of training and test points.

\begin{align}
\label{eq:joint_posterior}
  f_*|X_*,X,f \sim \mathcal{N} (K(X_*,X) K(X_,X)^{-1} \mathbf{f},\: K(X_*,X_*) - K(X_*,X) K(X,X)^{-1} K(X,X_*) )
\end{align}

Function values $f_*$ can be sampled from the joint posterior distribution by evaluating the mean and covariance matrix from \cite{eq:joint_posterior} \refeq{eq:joint_posterior}

\paragraph{Prediction using Noisy Observations}



\section{Classification}

For classification models, where the targets are discrete class labels, the Gaussian likelihood is inappropriate.
In the classification case the likelihood is non-Gaussian but the posterior process can be \emph{approximated} by a GP.

\subsection{Classification Problems}
Discussing joint probability $p(y, x)$, where $y$ denotes the class label. Using Bayes’ theorem
this joint probability can be decomposed either as $p(y)p(x|y)$ or as $p(x)p(y|x)$. This gives rise to two different approaches:
\emph{generative}: model the class-conditional probability $p(x|y)$ and poior $p(y)$ for each class, and then compute the posterior
\begin{align}
  p(y|\mathbf{x}) = { p(y)p(\mathbf{x}|y) \over p(\mathbf{x})} = {p(y)p(\mathbf{x}|y) \over \sum_C p(C_k)p(\mathbf{x}|C_k) }
\end{align}
\emph{discriminative}: model $p(y|x)$ directly.
For \textbf{generative} case, model the class-conditional densities with Gaussians: $p(x|C_k) = N(\mu_k, \Sigma_k)$
For \textbf{discriminative} case, model the class-conditional densities with \emph{linear logistic regression}: $p(x|C_k) = \lambda(\mathbf{x}^T\mathbf{w})$, $\lambda(x) = {1 \over 1+exp(-x)} $

\subsubsection{Decision Theory for Classification}
Predictive probabilities $p(y_* |x_*)$ for a test input $x_*$,

\subsection{Gaussian Process Classification}
For binary classification the basic idea behind Gaussian process prediction:
place a GP prior over the \emph{latent function} $f(x)$ and then squash this through the logistic function to obtain a prior on $\pi(x) \gets  p(y=+1|x) = \sigma(f(x))$.

Inference is naturally divided into two steps:
1. computing the distribution of the latent variable corresponding to a test case
\begin{align}
  p(f_*|X,y,x_*) = \int p(f_*|X,x_*,f)p(f|X,y) df
\end{align}
$p(f|X,y) = p(y|f)p(f|X)/p(y|X)$

2. using this distribution over the latent $f_*$ to produce a probabilistic prediction
\begin{align}
  \overline{\pi}_* \gets p(y_*=+1|X,y,x_*) = \int \sigma(f_*)p(f_*|X,y,x_*) df_*
\end{align}


\subsection{The Laplace Approximation for the Binary GP Classifier}


\end{document}
