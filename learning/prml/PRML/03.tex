\documentclass[a4paper]{article}
\usepackage{xeCJK}
\setCJKmainfont{STKaiti}
\usepackage{geometry}
\usepackage{mathtools}
\usepackage[sc,osf,slantedGreek]{mathpazo}

\begin{document}
\title{Bayesian Linear Regression}
\author{}
\maketitle



通过将一组输入变量的非线性进行线性组合,得到基函数。这样的模型是参数的线性函数。
by taking linear combinations of a fixed set of nonlinear functions of the input variables
Objective:
给输入集$\{x_n\}$及对应的目标$\{t_n\}$ey,预测新的$x$对应目标$t$的值.

\section{Linear Basis Function Models}
将输入变量的固定非线性函数进行线性组合:(得到参数$w_0,...,w_{M-1}$的线性函数),$w_0$称为偏置参数。
\begin{align*}
  y(\mathbf{x},\mathbf{w}) = w_0 + \sum_{n=1}^{M-1} w_j\phi_j(\mathbf{x})\\
  y(\mathbf{x},\mathbf{w}) = w^T \boldsymbol{\phi}^T(\mathbf{x})
\end{align*}
通过引入非线性基函数,我们能够让$y(\mathbf{x},\mathbf{w})$成为输入$x$的非线性函数,但上式被称为线性模型因为这个函数是参数$w$的线性函数。
基函数的选择:
\begin{align*}
  \phi_j(x) &= x^j &\text{ polynomial basis function}\\
  \phi_j(x) &= exp(-{(x-\mu_j)^2 \over 2s^2 }) &\text{ Gaussian basis function}\\
  \phi_j(x) &= \sigma(-{(x-\mu_j) \over s }) &\text{  sigmoid basis function}\\
  \sigma(a) &= {1 \over 1+exp(-a)}
\end{align*}

\subsection{Maximum likelihood and least squares}
输入集$X = \{x_1,...,x_N\}$, target $\{t_n\}$, likelihood function as bellow:
\begin{align*}
  p(\mathbf{x}|\mathbf{X},\mathbf{w},\beta) &= \prod_{n=1}^N \mathcal{N}(t_n|w^T\phi(\mathbf{x}_n), \beta^{-1})\\
\mathrm{ln} p(\mathbf{x}|\mathbf{w},\beta) &= \sum_{n=1}^N \mathrm{ln} \mathcal{N}(t_n|w^T\phi(\mathbf{x}_n), \beta^{-1}) \\
& = {N \over 2} \mathrm{ln} \beta - {N \over 2} \mathrm{ln} (2\pi) - \beta E_D(w)\\
E_D(w) &= {1 \over 2} \sum_{n=1}^N \{t_n -w^T\phi(x_n)\}^2\\
\end{align*}
\begin{align*}
\nabla \mathrm{ln} p(\mathbf{t}|w,\beta) &= \beta \sum_{n=1}^N \{t_n -w^T\phi(x_n)\} \phi(x_n)^T\\
w_{ML} &= (\Phi^T\Phi)^{-1}\Phi^Tt
\end{align*}
which are known as the \textit{normal equations} for the least squares problem. Here $\Phi$ is an $N \times M$ matrix, called \textit{design matrix}. whose elements are given by $\Phi_{nj} = \phi_j(x_n)$
\begin{align}
  \Phi =
  \begin{pmatrix}
    \phi_0(x_1) & \phi_1(x_1) & \cdots & \phi_{M-1}(x_1)\\
    \phi_0(x_2) & \phi_1(x_2) & \cdots & \phi_{M-1}(x_2)\\
    \vdots & \vdots & \ddots & \vdots\\
    \phi_0(x_N) & \phi_1(x_N) & \cdots & \phi_{M-1}(x_N)
  \end{pmatrix}
\end{align}

\subsection{Sequential learning}
\label{sec:3.1.3}

We can obtain a sequential learning algorithm by applying the technique of \textit{stochastic gradient descent}, also known as \textit{sequential gradient descent}, as follows
\begin{align*}
  w^{(\tau+1)} &= w^{(\tau)} + \eta \nabla E\\
  w^{(\tau+1)} &= w^{(\tau)} + \eta (t_n - w^{(\tau)^T}\phi_n) \phi_n
\end{align*}
$\eta$ is a learning rate parameter. $\phi_n = \phi(x_n)$. This is known as least-mean-squares or the LMS algorithm.

\subsection{Regularized least squares}
\label{sec:3.1.4}

Introduce the idea of adding a regularization term to an error function in order to control over-fitting
\begin{align}
  E_D(w) + \lambda E_w(w)
\end{align}

\section{The Bias-Variance Decomposition}
\label{sec:3.2}

A popular choice of loss function is squared loss functon, for which the optimal prediction is given by the conditional expectation, given by
\begin{align}
  h(x) = \mathbb{E}[t|x] = \int t p(t|x) dt
\end{align}
it is worth distinguishing between the \textit{squared loss function} arising from decision theory and the \textit{sum-of-squares error function} that arose in the maximum likelihood estimation of model parameters.
expected \emph{squared loss} can be written in the form
\begin{align}
  \mathbb{E}{L} = \int \{y(x) - h(x)\}^2 p(x) \mathrm{d}x + \iint \{h(x) -t\}^2 p(x,t)\mathrm{d}x \mathrm{d}t
\end{align}
The integrand of the first term:
\begin{align}
  \{y(x;D) - h(x)\}^2\\
  \{y(x|D) - \mathbb{E}_D[y(x;D)] + \mathbb{E}_D[y(x;D)] - h(x)\}^2
\end{align}
\begin{align*}
\mathbb{E}_D[ \{y(x;D) - h(x)\}^2 ] &= (bias)^2 + variance + noise \text{ (expected loss)}\\
(bias)^2 &= \int \{ \mathbb{E}_D[y(x;D)] - h(x) \}^2 p(x) dx\\
variance &= \int \mathbb{E}_D[\{ y(x;D) - \mathbb{E}_D[y(x;D)] \}^2 ] p(x) dx\\
noise &= \iint {h(x) - t}^2 p(x,t) \mathrm{d}x \mathrm{d}t
\end{align*}
squared \textit{bias}: 所有数据集的平均预测与预期的回归函数之间的差异
goal is to minimize the expected loss, which we have decomposed into the sum of a (squared) bias, a variance, and a constant noise term.
As we shall see, there is a trade-off between bias and variance, with very flexible models having low bias and high variance, and relatively rigid models having high bias and low variance.
The model with the optimal predictive capability is the one that leads to the best balance between bias and variance

\section{Bayesian Linear Regression}
\label{sec:3.3}


\subsection{Parameter distribution}
conjugate prior over the model parameters $ \mathbf{w} $
\[
p(\mathbf{w}) = \mathcal{N}(\mathbf{w} | \mathbf{m}_0, \mathbf{S}_0)
\]
having mean $ \mathbf{m}_0 $ and convariance $ \mathbf{S}_0 $
posterior distribution
\begin{align}
p(\mathbf{w}|\mathbf{t}) = \mathcal{N}(\mathbf{w} | \mathbf{m}_N, \mathbf{S}_N)
\\
\mathbf{m}_N = \mathbf{S}_N(\mathbf{S}_0^{-1} \mathbf{m}_0 + \beta \Phi^T \mathbf{t}))
\\
\mathbf{S}_N^{-1} = \mathbf{S}_0^{-1} + \beta \Phi^T \Phi
\end{align}

Specifically, we consider a zero-mean isotropic Gaussian governed by a single precision parameter $\alpha$ so that

\begin{align}
p(\mathbf{w}|\alpha) = \mathcal{N}(\mathbf{w} | \mathbf{0},
  \alpha^{-1} \mathit{\mathbf{I}})
\\
\mathbf{m}_N = \beta \mathbf{S}_N \Phi^T \mathbf{t}
\\
\mathbf{S}_N^{-1} =  \alpha \mathit{\mathbf{I}} + \beta \Phi^T \Phi
\end{align}

贝叶斯学习的顺序本质,当新数据点被观测到时,当前后验分布变成了先验分布。

\subsection{Predictive distribution}

\textit{predictive distribution}
\[
p(t | \mathbf{t}, \alpha, \beta) = \int p(t | \mathbf{w},
\beta)p(\mathbf(w)|\mathbf{t}, \alpha, \beta) d\mathbf{w}
\]
$ \mathbf{t} $ is the vector of target values from the training set,

\[
p(t | \mathbf{x}, \mathbf{t}, \alpha, \beta) = \mathcal{N}(t |
\mathbf{m}_N^T\Phi(\mathbf{x}),\sigma_N^2(\mathbf{x}))
\]

the variance $ \sigma_N^2(\mathbf{x}) $ of the predictive distribution is given by
\[
\sigma_N^2(\mathbf{x}) = \frac{1}{\beta} + \Phi(\mathbf{x})^T\mathbf{M}_N\Phi(\mathbf{x})
\]

\subsection{Equivalent kernel}
predictive mean
\begin{align*}
y(\mathbf{x},\mathbf{w}) = w^T \boldsymbol{\phi}^T(\mathbf{x})\\
\mathbf{m}_N = \beta \mathbf{S}_N \boldsymbol{\Phi}^T \mathbf{t}\\
S_N^{-1} = \alpha \mathbf{I} + \beta\Phi^T\Phi\\
\Rightarrow\\
y(\mathbf{x},\mathbf{m}_N) = \mathbf{m}_N^T \Phi(\mathbf{x}) =
\displaystyle\sum_{n=1}^{N} \beta \Phi(\mathbf{x})^T \mathcal{S}_N \Phi(\mathbf{x_n}) t_n\\
y(\mathbf{x},\mathbf{m}_N) = \sum k(\mathbf{x},\mathbf{x}_n)t_n\\
k(\mathbf{x},\mathbf{x}') = \beta\phi(\mathbf{x})^T\mathbf{S}_N\phi(x')
\end{align*}
the mean of the predictive distribution at a point
$ \mathbf{x} $ is given by a linear combination of the training set target
variables $ t_n $

\begin{align}
y(\mathbf{x},\mathbf{m}_N) =
\displaystyle\sum_{n=1}^{N} k(\mathbf{x},\mathbf{x}_n) t_n
\\
 k(\mathbf{x},\mathbf{x}') = \beta \phi(\mathbf{x})^T \mathbf{S}_N \phi(\mathbf{x'})
\end{align}
is known as the \textit{smoother matrix} or the \textit{equivalent
  kernel}. Regression functions, such
as this, which make predictions by taking linear combinations of the training set
target values are known as \textit{linear smoothers}. Intuitively, it seems reasonable
that we should weight local evidence more strongly than distant evidence.

Instead of introducing a set of basis functions, which implicitly determines an equivalent kernel, we can instead define a localized kernel directly and use this to make predictions for new input vectors $x$, given the observed training set. This leads to a practical framework for regression (and classification) called \textit{Gaussian processes}.
We have seen that the effective kernel defines the weights by which the training set target values are combined in order to make a prediction at a new value of $x$, and it can be shown that these weights sum to one, in other words
\begin{align}
\displaystyle\sum_{n=1}^{N} k(\mathbf{x},\mathbf{x}_n) = 1
\end{align}
Finally, we note that the equivalent kernel (3.62) satisfies an
important property shared by kernel functions in general, namely that
it can be expressed in the form an inner product with respect to a
vector $ \psi(\mathbf{x} $ of nonlinear functions, so that
\begin{align}
k(\mathbf{x},\mathbf{z}) = \psi(\mathbf{x})^T\psi(\mathbf{x})
\\
\psi(\mathbf{x}) = \beta^{1/2} \mathbf{S}_N^{1/2} \phi(\mathbf{x})
\end{align}


\end{document}

%%% Local Variables:
%%% mode: latex
%%% TeX-master: t
%%% End:
