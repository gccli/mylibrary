\documentclass{article}
\usepackage{xeCJK}
%\setCJKmainfont{Microsoft YaHei}
\setCJKmainfont[BoldFont={WenQuanYi Zen Hei}, ItalicFont={WenQuanYi Zen Hei}]{WenQuanYi Zen Hei Mono}
%\setCJKfamilyfont{hei}{WenQuanYi Zen Hei}
%\setCJKmainfont{SimSun}
%\setCJKfamilyfont{song}{SimSun}

\begin{document}
\title{学习LaTeX}

\textbf{TeX格式}

最基本的TeX程序只是由一些很原始的命令组成,它们可以完成简单的排版操作和程序设计功能。然而,TeX也允许用这些原始命令定义一些更复杂的高级命令。这样就可以利用低级的块结构,形成一个用户界面相当友好的环境。

在处理器运行期间,该程序首先读取所谓的格式文件,其中包含各种以原始语言写成的高级命令,也包含分割单词的连字号安排模式。接着处理程序就处理源文件,其中包含要处理的真正文本,以及在格式文件中已定义了的格式命令。

创建新格式是一件需要由具有丰富知识程序员来做的事情。把定义写到一个源文件中,这个文件接着被一个名叫initex的特殊版本的TeX程序处理。它采用一种紧凑的方式存贮这些新格式,这样就可以被通常TeX程序很快地读取。

\textbf{Plain TeX}

Knuth设计了一个名叫Plain TeX的基本格式,以与低层次的原始TeX呼应。这种格式是用TeX处理文本时相当基本的部分,以致于我们有时都分不清到底哪条指令是真正的处理程序TeX的原始命令,哪条是Plain TeX格式的。大多数声称只使用TeX的人,实际上指的是只用Plain TeX。

Plain TeX也是其它格式的基础,这进一步加深了很多人认为TeX和Plain TeX是同一事物的印象。

\textbf{LaTeX}

Plain TeX的重点还只是在于如何排版的层次上,而不是从一位作者的观点出发。对它的深层功能的进一步发掘,需要相当丰富的编程技巧。因此它的应用就局限于高级排版和程序设计人员。

Leslie Lamport开发的LaTeX是当今世界上最流行和使用最为广泛的TeX格式。它构筑在Plain TeX的基础之上,并加进了很多的功能以使得使用者可以更为方便的利用TeX的强大功能。使用LaTeX基本上不需要使用者自己设计命令和宏等,因为LaTeX已经替你做好了。因此,即使使用者并不是很了解TeX,也可以在短短的时间内生成高质量的文档。对于生成复杂的数学公式,LaTeX表现的更为出色。

\textbf{LaTeX2e}

LaTeX自从二十世纪八十年代初问世以来,也在不断的发展。最初的正式版本为2.09,在经过几年的发展之后,许多新的功能,机制被引入到LaTeX中。在享受这些新功能带来的便利的同时,它所伴随的副作用也开始显现,这就是不兼容性。标准的LaTeX 2.09,引入了“新字体选择框架”(NFSS)的LaTeX,SLiTeX,AMSLaTeX等等,相互之间并不兼容。这给使用者和维护者都带来很大的麻烦。

为结束这中糟糕的状况,Frank Mittelbach 等人成立了LaTeX3项目小组,目标是建立一个最优的,有效的,统一的,标准的命令集合。即得到LaTeX的一个新版本3。这是一个长期目标,向这个目标迈出第一步就是在1994年发布的LaTeX2e。LaTeX2e采用了NFSS作为标准,加入了很多新的功能,同时还兼容旧的LaTeX 2.09。LaTeX2e每6个月更新一次,修正发现的错误并加入一些新的功能。在LaTeX3最终完成之前,LaTeX2e将是标准的LaTeX版本。

\paragraph{中文输入}

texdoc xecjk


\begin{thebibliography}{99}
\bibitem[1]{latex tutorial} $http://cs2.swfc.edu.cn/~wx672/lecture_notes/linux/latex/latex_tutorial.html $
\end{thebibliography}

\end{document}


%%% Local Variables:
%%% mode: latex
%%% TeX-master: t
%%% End:
