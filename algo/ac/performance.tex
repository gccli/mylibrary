\documentclass{article}
\usepackage{xeCJK}
\setCJKmainfont[BoldFont={WenQuanYi Zen Hei}, ItalicFont={WenQuanYi Zen Hei}]{WenQuanYi Zen Hei Mono}
\usepackage{geometry}
\usepackage[table]{xcolor}
\usepackage{listings}
\usepackage{color}
\usepackage{graphicx}
\usepackage{multirow}
\usepackage{alltt}
\usepackage{tabularx}
\usepackage{array}
\usepackage{dcolumn}
\usepackage[linkcolor=red]{hyperref}
%\inputencoding{utf8}
% Back to the UTF-8 encoding.
% ...


\lstdefinestyle{customc}{
  belowcaptionskip=1\baselineskip,
  breaklines=true,
  frame=L,
  xleftmargin=\parindent,
  language=C,
  showstringspaces=false,
  basicstyle=\footnotesize\ttfamily,
  keywordstyle=\bfseries\color{green!40!black},
  commentstyle=\itshape\color{purple!40!black},
  identifierstyle=\color{blue},
  stringstyle=\color{orange},
}
\lstset{escapechar=@,style=customc}

%\usepackage[hidelinks,linkcolor=red]{hyperref}
\begin{document}
\title{Multi-Pattern Search Algorithm comparison}
\author{}
\maketitle


\subparagraph{Description of Aho-Corasick algorithm variants}


\begin{description}
\item[AC\_STD] \hfill \\
  Standard DFA, based on this \hyperlink{effic}{paper}. Using \textit{state table}  \textit{state transition} storage, each entry in state table had a vector of transitions for state, a failure state  and match pattern list.
\begin{lstlisting}
typedef struct  {
    int      NextState[ ALPHABET_SIZE ];
    int      FailState;
    ACSM_PATTERN *MatchList;
}ACSM_STATETABLE;
\end{lstlisting}
\item[AC\_FULL] \hfill \\
  Full matrix, broken state table above into state transition table, an array of per state matching pattern lists, and a separate failure pointer list for each state for the NFA.
\begin{lstlisting}
typedef struct trans_node_s {
  acstate_t    key;
  acstate_t    next_state;
  struct trans_node_s * next; /* next transition for this state */
} trans_node_t;

typedef struct {
    int acsmMaxStates; // depend on the total bytes of keywords
    int acsmNumStates; // <= acsmMaxState

    ACSM_PATTERN2    * acsmPatterns;

    //per-state failure state, used for build NFA
    acstate_t        * acsmFailState;

    // 每个状态对应一个MatchList
    ACSM_PATTERN2   ** acsmMatchList;

    /**
     * list of transitions in each state, this is used to build the nfa and
     * dfa after construction we convert to sparse or full format matrix
     * and free the transition lists
     */
// 基于链表的状态跃迁表,用于创建NFA和DFA,当构建完成,将其转化为full format matrix然后将内存释放
    trans_node_t ** acsmTransTable;

// Alloc a separate state transition table == in state 's' due to event 'k', transition to 'next' state
    acstate_t ** acsmNextState;
...
} ACSM_STRUCT2;
\end{lstlisting}
\item[AC\_FULLQ] \hfill \\
  ditto, but matching states are queued
\item[AC\_SPARSE] \hfill \\
  Sparse matrix
\item[AC\_BANDED] \hfill \\
  Banded matrix
\item[AC\_SPARSEBANDS] \hfill \\
  Sparse-Banded matrix
\end{description}



\subparagraph{Test Result}

\textbf{Note}:
\begin{enumerate}
  \item All keyword generated randomly, the keyword includes english and chinese
  \item The standard AC implemented by me currently only support case sensitive
  \item The memory show in table 1 just the total allocate size, not the actual page size allocated by kernel.
\end{enumerate}


\begin{table}[h]
\caption{\textsc{performance}}
\begin{tabular}{ |l|p{2cm}|p{2cm}|p{2cm}|p{2cm}|p{2cm}|p{2cm}| }
\hline\hline
& \multicolumn{3}{|c|}{pattern count 120}  &
  \multicolumn{3}{|c|}{pattern count 240} \\
\cline{2-7}
\textbf{Algorithm} & \textbf{Compile Time (s)}  & \textbf{Search Time (s)} & \textbf{Memory (kB)} & \textbf{Compile Time (s)} & \textbf{Search Time (s)} & \textbf{Memory (kB)} \\
\hline\hline
Standard      & 0.0093094 & 4.4257226 & 1083.24 & 0.0185856 & 4.499131 & 2141.07\\
Full Matrix   & 0.0274246 & 5.4931084 & 949.76 & 0.0544766 & 5.622883 & 1837.88\\
Full Queueed  & 0.032692  & 5.5069188 & 949.76 & 0.0479468 & 6.9196432 & 1837.88\\
Sparse Matrix & 0.0299582 & 6.0480434 & 233.22 & 0.0479468 & 6.9196432 & 495.41\\
Banded Matrix & 0.0299582 & 6.0480434 & 652.48 & 0.0479468 & 6.9196432 & 1206.72\\
Sparse Banded & 0.0300112 & 6.43706 & 180.74 & 0.0458898 & 6.4687386 & 339.4\\
\hline
\end{tabular}
\end{table}

\begin{figure}[h]
\caption{\textsc{Memory allocate}}
\includegraphics[scale=0.8]{/root/share/x.PNG}
\end{figure}

\begin{figure}[h]
\caption{\textsc{Search Time}}
\includegraphics[scale=0.8]{/root/share/y.PNG}
\end{figure}

\begin{figure}[h]
\caption{\textsc{VmSzie}}
\includegraphics[scale=0.8]{/root/share/z.PNG}
\end{figure}


\subparagraph{Summary}

From the figure 1, we can see that, all implementment of AC in acsmx2.c has remarkable drop compare with $AC\_STD$, but the memory just total allocate size, can not indicate real linux memory allocate. We known that, linux memory allocation is 4k aligned (the page size is 4k). see figure 3
By compress state table(by the total bytes of keywords), we can still reduce memory usage in $AC\_STD$. In addition, the $AC\_STD$ has best performance and straightforward implement, simple and fast. In keyword size not more than 200, we could just use $AC\_STD$


\subparagraph{Reference}
\begin{list}{}{}
  \item \hypertarget{effic}{[1]} \href{http://cr.yp.to/bib/1975/aho.pdf}{\textit{Efficient String matching: An Aid to Bibliographic Search}} \\
  \item \hypertarget{snort}{[2]} \href{http://citeseerx.ist.psu.edu/viewdoc/download?doi=10.1.1.103.4663&rep=rep1&type=pdf}{\textit{Optimizing Pattern Matching for Intrusion Detection}}
\end{list}

\end{document}

%%% Local Variables:
%%% mode: latex
%%% TeX-master: t
%%% End:
