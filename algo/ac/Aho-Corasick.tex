\documentclass{article}
\usepackage{xeCJK}
\setCJKmainfont[BoldFont={WenQuanYi Zen Hei}, ItalicFont={WenQuanYi Zen Hei}]{WenQuanYi Zen Hei Mono}
\usepackage{geometry}
\usepackage{graphicx}
\usepackage{mathtools}
\usepackage[sc,osf,slantedGreek]{mathpazo}
\usepackage{color}
\usepackage{listings}
\lstdefinestyle{customc}{
  belowcaptionskip=1\baselineskip,
  breaklines=true,
  frame=L,
  xleftmargin=\parindent,
  language=C,
  showstringspaces=false,
  basicstyle=\footnotesize\ttfamily,
  keywordstyle=\bfseries\color{green},
  commentstyle=\itshape\color{purple!40!black},
  commentstyle=\color{red},
  identifierstyle=\color{blue},
  stringstyle=\color{orange},
}
\lstset{escapechar=@,style=customc}

%\usepackage[hidelinks,linkcolor=red]{hyperref}

\begin{document}
\title{Aho-Corasick Automaton Introduction}
\author{}
\maketitle

\section{Introduction}
The behaviour of the pattern matching machine is dictated by three functions: a goto function $g$, a failure function $f$, and an output function $output$

\begin{figure}[h]
  \centering
\caption{\textsc{Pattern matching machine}}
\includegraphics[scale=0.6]{/root/share/ac.png}
\end{figure}

现在让我们来定义这三个函数,$goto$函数将一个状态和一个输入映射到另一个状态或\emph{fail},称之为状态跃迁(\textit{goto transition})。$failure$函数将一个状态映射到另一个状态,当$goto$函数返回\emph{fail}时调用(consulted)。$output$函数
\begin{align*}
goto    &:= (s,a) \mapsto
\begin{cases}
 s' \\
fail & \quad \text{ (the absense of an arrow)}
\end{cases}\\
f       &:= (s) \mapsto s'\\
output  &:= (s) \mapsto \{patterns\} \quad \text{ (associate a set of keywords or empty)}\\
\end{align*}

\section{Implementation}
AC算法的实现有多种变体,基本在snort的源代码中都能找到,以下描述几种最基本的变体。
\begin{description}
\item[AC\_STD] \hfill \\
标准AC实现,即基于AC最初的论文\ref{1}实现的DFA,而实现的方法又决定着状态机的性能。我参考snort中acsmx.c(即AC实现第一版),通过以下结构存储AC状态机的状态跃迁表:
\begin{lstlisting}
typedef struct  {
    int      NextState[ ALPHABET_SIZE ];
    int      FailState;
    ACSM_PATTERN *MatchList;
}ACSM_STATETABLE;
\end{lstlisting}
\item[AC\_FULL] \hfill \\
全矩阵(\textit{full matrix})实现对上述实现的存储方案进行了优化,将上面的StateTable分解为每状态的状态跃迁表(其结构为单向链表),每状态匹配表(MatchList)和用于NFA的每状态矢效指针,如下结构所示:\\
将构建完成后的NFA或DFA存储在acsmTransTable(图2)中,全矩阵则是将acsmTransTable转化为acsmNextState。以下四种实现都是基于此实现,将全矩阵转化为Sparse或Banded…
\begin{lstlisting}
typedef struct trans_node_s {
  acstate_t    key; // input symbol
  acstate_t    next_state;
  struct trans_node_s * next; /* next transition for this state */
} trans_node_t;

typedef struct {
    int acsmMaxStates; // 可能的最大状态
    int acsmNumStates; // 有效状态总数,active state
    ACSM_PATTERN2    * acsmPatterns;//关键字列表
//per-state failure state, used for build NFA
    acstate_t        * acsmFailState;

// 每个状态对应一个MatchList
    ACSM_PATTERN2   ** acsmMatchList;
// 基于链表的状态跃迁表,用于创建NFA和DFA,当构建完成,将其转化为full format matrix然后将内存释放
    trans_node_t ** acsmTransTable;
// 单独的状态跃迁表
    acstate_t ** acsmNextState;
...
} ACSM_STRUCT2;
\end{lstlisting}
\item[AC\_FULLQ] \hfill \\
  ditto, but matching states are queued
\item[AC\_SPARSE] \hfill \\
  Sparse matrix
\item[AC\_BANDED] \hfill \\
  Banded matrix
\item[AC\_SPARSEBANDS] \hfill \\
  Sparse-Banded matrix
\end{description}

\section{Principle}

\subsection{Construct goto}
为了理解Snort对标准实现的改进及改进所用的数据结构,我们以$AC^{FULL}$为例,构建goto function

\begin{figure}[h]
  \centering
\caption{\textsc{goto corresponding to $AC^{FULL}$}}
\begin{verbatim}
-------- Add "he", TransTable --------       -------- Add "she", TransTable --------
  0: (H->1)                                    0: (S->3) -> (H->1)
  1: (E->2)                                    1: (E->2)
                                               2:
                                               3: (H->4)
                                               4: (E->5)
-------- Add "his", TransTable --------      -------- Add "hers", TransTable --------
  0: (S->3) -> (H->1)                          0: (S->3) -> (H->1)
  1: (I->6) -> (E->2)                          1: (I->6) -> (E->2)
  2:                                           2: (R->8)
  3: (H->4)                                    3: (H->4)
  4: (E->5)                                    4: (E->5)
  5:                                           5:
  6: (S->7)                                    6: (S->7)
                                               7:
                                               8: (S->9)
\end{verbatim}
\end{figure}

\subsection{Build NFA}
\begin{enumerate}
  \item Make $f(s) = 0$ for all $s$ of depth 1
  \item 假设所有深度小于$d$的failure状态都已计算过了,则深度$d$的failure状态通过深度为$d-1$ goto的nonfail值确定。其中$d-1$的状态$r$,过程如下:
 \begin{enumerate}
    \item $g(r,a) == fail$ for all $a$, do nothing
    \item 否则当$g(r, a) = s$,$s$不为fail
      \begin{enumerate}
        \item state = f(r)
        \item while(g(state, a) != fail) state = f(state)
        \item set f(s) = g(state,a)
      \end{enumerate}
 \end{enumerate}
\end{enumerate}

\begin{lstlisting}
foreach(allsymbols as a)
  if ((s = goto(r,a)) != fail)
    state = f(r);
    while(goto(state,a) != fail)
      state = goto(state,a);
    f(s) = state;
\end{lstlisting}

\textbf{Examples}:
\begin{list}{}{}
  \item  f(1) = f(3) = 0
  \item  state 2: g(1,e) = 2(s); state=f(1)=0, while( (g(0,e) = 0) != fail) so f(s=2) = g(0,e) = 0
  \item  state 4: g(3,h) = 4(s); state=f(3)=0; while( (g(0,h) = 1) != fail) so f(s=4) = g(0,h) = 1
\end{list}
至此,我们生成了一个NFA,如下图所示:
\begin{figure}[h]
  \centering
  \caption{final fail function}
\includegraphics[scale=0.6]{/root/share/failure.png}
\end{figure}
\subsection{Convert NFA to DFA}

通过以上算法生成的failure function在以下情况不是最优的:
设模式匹配状态机M,我们有$g(4,e) = 5$
如果M当前状态为4并且当前输入为$a_i$非$e$,则M将进入状态$f(4)=1$。因为M已经知道$a_i \ne e$,M无需考虑goto在状态为1输入为$e$时的值$g(1,e)$。事实上,如果关键字”his”不在,M可以直接的从状态4进入状态0,跳过多余的中间转移到状态1。\\
为了完全避免\textit{failure transitions},我们定义DFA版本的算法:
\textbf{DFA}(deterministic finite automaton)包含一个有限的状态集$S$和一个\textit{next move} function $\delta$ s.t. $\forall a, \delta(s,a) \in S$. 也就是说,DFA使每一输入精确的与一个状态对应。\\
我们可以通过以下算法将NFA中的goto和failure生成\textit{next move function} $\delta$:
\begin{lstlisting}
queue = empty
foreach(symbols as a)
  delta(0,a) = g(0,a)
  if g(0,a) != 0 then queue.push(g(0,a))

while !queue.empty()
  r = queue.front(); queue.pop();
  foreach(symbols as a)
    if(g(r,a) = s != fail
      queue.push(s)
      delta{r,a} = s
    else
      delta(r,a) = delta(f(r),a)
\end{lstlisting}

DFA和NFA最大的区别在于DFA能够\textbf{精确的}将每一个输入映射在有限集$S$中,NFA则需中间状态的过渡。

\begin{thebibliography}{99}
\bibitem[1]{1} Efficient String matching: An Aid to Bibliographic Search]
\bibitem[2]{2} Optimizing Pattern Matching for Intrusion Detection
\end{thebibliography}

\end{document}

%%% Local Variables:
%%% mode: latex
%%% TeX-master: t
%%% End:
